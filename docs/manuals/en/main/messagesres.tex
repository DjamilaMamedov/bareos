
\chapter{Messages Configuration}
% used as link from developer guide
\label{MessagesChapter}
\label{ResourceMessages}
\index[general]{Resource!Messages}
\index[general]{Messages Resource}

\section{Messages Resource}

The Messages resource defines how messages are to be handled and destinations
to which they should be sent.

Even though each daemon has a full message handler, within the \bareosFd and
the \bareosSd, you will normally choose to send all the appropriate
messages back to the \bareosDir.  This permits all the messages associated with
a single Job to be combined in the Director and sent as a single email message
to the user, or logged together in a single file.

Each message that Bareos generates (i.e. that each daemon generates) has an
associated type such as INFO, WARNING, ERROR, FATAL, etc. Using the message
resource, you can specify which message types you wish to see and where they
should be sent. In addition, a message may be sent to multiple destinations.
For example, you may want all error messages both logged as well as sent to
you in an email. By defining multiple messages resources, you can have
different message handling for each type of Job (e.g. Full backups versus
Incremental backups).

In general, messages are attached to a Job and are included in the Job report.
There are some rare cases, where this is not possible, e.g. when no job is
running, or if a communications error occurs between a daemon and the
director. In those cases, the message may remain in the system, and should be
flushed at the end of the next Job.

The records contained in a Messages resource consist of a {\bf destination}
specification followed by a list of {\bf message-types} in the format:
\index[dir]{Messages!destination}

\begin{description}
\item [destination = message-type1, message-type2, message-type3, ...  ]
\end{description}

or for those destinations that need and address specification (e.g. email):

\begin{description}

\item [destination = address = message-type1, message-type2, message-type3, ...] \hfill \\

where

\begin{description}
    \item [destination] is one of a predefined set of keywords that define
where the message is to be sent (\linkResourceDirective{Dir}{Messages}{Append}, \linkResourceDirective{Dir}{Messages}{Console}, \linkResourceDirective{Dir}{Messages}{File}, \linkResourceDirective{Dir}{Messages}{Mail}, ...), 
    \item [address] varies according to the {\bf destination} keyword, but
is typically an email address or a filename,
    \item [\ilink{message-type}{MessageTypes}] is one of a predefined set of keywords that define the type of
message generated by Bareos: {\bf ERROR}, {\bf WARNING}, {\bf FATAL},
...
\end{description}

\end{description}

\input{autogenerated/bareos-dir-resource-messages-table.tex}
\input{resource-messages-definitions.tex}
\input{autogenerated/bareos-dir-resource-messages-description.tex}

\section{Message Types}
\label{MessageTypes}

For any destination, the {\bf message-type} field is a comma separated
list of the following types or classes of messages:

\begin{description}

\item [info] \hfill \\
\index[general]{Messages!type!info}
General information messages.

\item [warning] \hfill \\
\index[general]{Messages!type!warning}
Warning messages. Generally this is some  unusual condition but not expected
to be serious.

\item [error] \hfill \\
\index[general]{Messages!type!error}
Non-fatal error messages. The job continues running.  Any error message should
be investigated as it means that something  went wrong.

\item [fatal] \hfill \\
\index[general]{Messages!type!fatal}
Fatal error messages. Fatal errors cause the  job to terminate.

\item [terminate] \hfill \\
\index[general]{Messages!type!terminate}
Message generated when the daemon shuts down.

\item [notsaved] \hfill \\
\index[general]{Messages!type!notsaved}
Files not saved because of some error.  Usually because the file cannot be
accessed (i.e. it does not  exist or is not mounted).

\item [skipped] \hfill \\
\index[general]{Messages!type!skipped}
Files that were skipped because of a user supplied option such as an
incremental backup or a file that matches an exclusion pattern.  This is
not considered an error condition such as the files listed for the {\bf
notsaved} type because the configuration file explicitly requests these
types of files to be skipped.  For example, any unchanged file during an
incremental backup, or any subdirectory if the no recursion option is
specified.

\item [mount] \hfill \\
\index[general]{Messages!type!mount}
Volume mount or intervention requests from the Storage daemon.  These
requests require a specific operator intervention for the job to
continue.

\item [restored] \hfill \\
\index[general]{Messages!type!restored}
The {\bf ls} style listing generated for each file restored is sent to
this message class.

\item [all] \hfill \\
\index[general]{Messages!type!all}
All message types.

\item [security] \hfill \\
\index[general]{Messages!type!security}
Security info/warning messages principally from unauthorized
connection attempts.

\item [alert] \hfill \\
\index[general]{Messages!type!alert}
Alert messages. These are messages generated by tape alerts.

\item [volmgmt] \hfill \\
\index[general]{Messages!type!volmgmt}
Volume management messages. Currently there are no volume management
messages generated.

\item [audit] \hfill \\
\index[general]{Messages!type!audit}
\index[general]{auditing}
Audit messages. Interacting with the Bareos Director will be audited.
Can be configured with in resource \linkResourceDirective{Dir}{Director}{Auditing}.

\end{description}

The following is an example of a valid Messages resource definition, where
all messages except files explicitly skipped or daemon termination messages
are sent by email to backupoperator@example.com.  In addition all mount messages
are sent to the operator (i.e.  emailed to backupoperator@example.com).  Finally
all messages other than explicitly skipped files and files saved are sent
to the console:

\begin{bconfig}{Message resource}
Messages {
  Name = Standard
  Mail = backupoperator@example.com = all, !skipped, !terminate
  Operator = backupoperator@example.com = mount
  Console = all, !skipped, !saved
}
\end{bconfig}

With the exception of the email address,
an example Director's Messages resource is as follows:

\begin{bconfig}{Message resource}
Messages {
  Name = Standard
  Mail Command = "/usr/sbin/bsmtp -h mail.example.com  -f \"\(Bareos\) %r\" -s \"Bareos: %t %e of %c %l\" %r"
  Operator Command = "/usr/sbin/bsmtp -h mail.example.com -f \"\(Bareos\) %r\" -s \"Bareos: Intervention needed for %j\" %r"
  Mail On Error = backupoperator@example.com = all, !skipped, !terminate
  Append = "/var/log/bareos/bareos.log" = all, !skipped, !terminate
  Operator = backupoperator@example.com = mount
  Console = all, !skipped, !saved
}
\end{bconfig}
